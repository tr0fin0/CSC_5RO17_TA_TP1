\documentclass[../CSC_5RO17_TA_TP1.tex]{subfiles}

\begin{document}
\subsection*{Question 3}

Homographie
\begin{equation}
    \begin{bmatrix}

    \end{bmatrix}
\end{equation}
la définition de la matrix A est donnée:
% definir matrix A
de cette façon, si on considére les mêmes points presents pour la question 2:
% definit pontos
la matrix A sera:
% escrever matrix A completa
de cette façon pour résoudre le problème le méthode de SVD a été utilisé, donnant le résultat suivant:
% incluir código
% incluir resultado


en applicant l'homographie sur l'image initiale nous avons le résultat suivant:
% image de comparação original vs DLT vs CV2
le résultat obtendu avec OpenCV est ajouté à la image au déçu comme réference d'un résultat idéale.
% ajouter code pour opencv


maintenant si on considère la normalisation des coordonnées le résultat serai le suivant:
% image de comparação original, DLT vs normalisé
les résultats normalisé sont % analyse

la normalisation a pour interet enleve la dependance de l estimation de la homographie de la taille de la image en normalisant ses dimensions avant d'appliquer le method.

de cette façon le erreur causé par la taille de la image serait-il reduit


% chatgpt
Reliability of the Result

How to evaluate the reliability of the homography?

One of the most common ways to assess the quality and reliability of the computed homography is by using a confidence index based on the reprojection error. Reprojection error measures how closely the projected points (via the homography) match the actual points in the image.

    Reprojection Error: After computing the homography matrix HH, you can reproject the original points using this matrix and compare them to the actual corresponding points. The smaller the reprojection error, the more reliable the homography.

    Singular Values: Another way to assess reliability is by examining the singular values from the SVD during the computation. If the smallest singular value is significantly smaller than the others, the system may be near-degenerate (indicating potential issues with point configuration or conditioning).

\end{document}
