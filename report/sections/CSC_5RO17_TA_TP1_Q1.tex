\documentclass[class=article, crop=false]{standalone}

\begin{document}
\subsection*{Question 1}
\begin{definition}
    Une homographie est une transformation projective entre deux plans qui conserve les lignes droites.
\end{definition}
\noindent En termes simples, cela signifie que si deux images d'une scène sont prises sous certaines conditions géométriques, la relation entre ces images peut être modélisée par une homographie.
\begin{example}
    Cas où la transformation est une homographie:
    \begin{enumerate}[noitemsep]
        \item images d'un même plan;
        \item images prises par une caméra tournant autour de son centre optique;
        \item transformation d'un plan à travers une projection 3D;
        \item stitching d'images;
    \end{enumerate}
\end{example}
\end{document}