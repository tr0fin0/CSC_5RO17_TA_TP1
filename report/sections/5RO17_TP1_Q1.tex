\documentclass[../5RO17_TP1.tex]{subfiles}

\begin{document}
\subsection{Question 1}
\begin{definition}
    Une \textbf{Homographie} $\mathbf{H}$ est une transformation projective entre deux plans $\pi_1$ et $\pi_2$ qui conserve les lignes droites.\\
    
    \noindent Une \textbf{Homographie} peut-être represente par l'équation suivant:
    \begin{equation}
        \boxed{
            \underbrace{
                \begin{bmatrix}
                x_2\\ y_2\\ 1
                \end{bmatrix}
            }_{\mathbf{x}_{2}}
            =
            \underbrace{
                \begin{bmatrix}
                    a_{11} & a_{12} & t_x\\
                    a_{21} & a_{22} & t_y\\
                    v_1    & v_2    & 1\\
                \end{bmatrix}
            }_{\text{Homographie},\;\mathbf{H}}
            \times
            \underbrace{
                \begin{bmatrix}
                    x_1\\ y_1\\ 1
                \end{bmatrix}
            }_{\mathbf{x}_{1}}
        }
        \quad
        \text{où}
        \quad
        \begin{cases}
            \mathbf{a} &= \begin{bmatrix}
                a_{11} & a_{12}\\
                a_{21} & a_{22}\\
            \end{bmatrix}\vspace{2mm}\\
            \mathbf{t} &= \begin{pmatrix} t_{x} & t_{y} \end{pmatrix}^{\intercal}\vspace{2mm}\\
            \mathbf{v} &= \begin{pmatrix} v_{1} & v_{2} \end{pmatrix}
        \end{cases}
    \end{equation}
    Où:
    \begin{enumerate}[noitemsep]
        \item $\mathbf{x}_{2}$, \textbf{Vecteur Destination}: coordonnées après l'homographie;
        \item $\mathbf{x}_{1}$, \textbf{Vecteur Source}: coordonnées avant l'homographie;
        \item $\mathbf{a}$, \textbf{Matrix d'Affinité};
        \item $\mathbf{t}$, \textbf{Vecteur de Translation};
        \item $\mathbf{v}$, \textbf{Vecteur de Relation};
    \end{enumerate}
\end{definition}
\begin{remark}
    Dans ce travail practique, une vecteur $\mathbf{x}_{i} = \begin{bmatrix} x_{i} & y_{i} & 1 \end{bmatrix}^{\intercal}$ sera construit à partir d'un point $\mathbf{p}_{i} = \begin{pmatrix} x_{i} & y_{i} \end{pmatrix}$.
\end{remark}
\noindent En termes simples, cela signifie que si deux images d'une scène sont prises sous certaines conditions géométriques, la relation entre ces images peut être modélisée par une \textbf{Homographie}.\\

\noindent Cas où la transformation est une homographie:
\begin{enumerate}[noitemsep, rightmargin=\leftmargin]
    \item images d'un même plan vu sous deux poses 3D différentes;
    \item images prises par une caméra tournant autour de son centre optique;
\end{enumerate}
\begin{remark}
    Les images destination et source devront avoir des données en commun pour que la transformation soit considérée comme une homographie.
\end{remark}
\end{document}
