\documentclass[../CSC_5RO17_TA_TP1.tex]{subfiles}

\begin{document}
\subsection{Question 2}
\noindent L'\textbf{homographie} $\mathbf{H}$ contient 8 variables inconnues. Ainsi, 4 correspondances de points arbitraires $\mathbf{p}{i{\pi_1}} = \begin{pmatrix}x_{1i} & y_{1i}\end{pmatrix} \in \pi_1$ et $\mathbf{p}{i{\pi_2}} = \begin{pmatrix}x_{2i} & y_{2i}\end{pmatrix} \in \pi_2$, non colinéaires, sont nécessaires pour estimer la matrice $\mathbf{H}$.\\

\noindent L'image originale \textit{Pompei.jpg} est considérée comme le plan $\pi_1$, et l'image \textit{Pompei.jpg} recadrée est considérée comme le plan $\pi_2$. Les paires de points choisies seront:
\begin{table}[H]
    \centering
    \begin{subtable}[H]{0.475\textwidth}
        \begin{tabular}{cll}
            \hline
            - & $\mathbf{p}_{i_{\pi_1}}$ & $\mathbf{p}_{i_{\pi_2}}$\\
            \hline
            1 & $\mathbf{p}_{1_{\pi_1}} = \begin{pmatrix}120 & 023\end{pmatrix}$ & $\mathbf{p}_{1_{\pi_2}} = \begin{pmatrix}083 & 000\end{pmatrix}$\\
            2 & $\mathbf{p}_{2_{\pi_1}} = \begin{pmatrix}073 & 287\end{pmatrix}$ & $\mathbf{p}_{2_{\pi_2}} = \begin{pmatrix}083 & 333\end{pmatrix}$\\
            3 & $\mathbf{p}_{3_{\pi_1}} = \begin{pmatrix}409 & 299\end{pmatrix}$ & $\mathbf{p}_{3_{\pi_2}} = \begin{pmatrix}416 & 333\end{pmatrix}$\\
            4 & $\mathbf{p}_{4_{\pi_1}} = \begin{pmatrix}386 & 022\end{pmatrix}$ & $\mathbf{p}_{4_{\pi_2}} = \begin{pmatrix}416 & 000\end{pmatrix}$\\
            \hline
        \end{tabular}
        \caption{coordonnées $\mathbf{p}_{i_{\pi_{j}}}$}
        \label{tab_homography_points}
    \end{subtable}
    \begin{subtable}[H]{0.475\textwidth}
        \begin{tabular}{cll}
            \hline
            - & $\mathbf{x}_{i_{\pi_1}}$ & $\mathbf{x}_{i_{\pi_2}}$\\
            \hline
            1 & $\mathbf{x}_{1_{\pi_1}} = \begin{pmatrix}120 & 023 & 1\end{pmatrix}^{\intercal}$ & $\mathbf{x}_{1_{\pi_2}} = \begin{pmatrix}083 & 000 & 1\end{pmatrix}^{\intercal}$\\
            2 & $\mathbf{x}_{2_{\pi_1}} = \begin{pmatrix}073 & 287 & 1\end{pmatrix}^{\intercal}$ & $\mathbf{x}_{2_{\pi_2}} = \begin{pmatrix}083 & 333 & 1\end{pmatrix}^{\intercal}$\\
            3 & $\mathbf{x}_{3_{\pi_1}} = \begin{pmatrix}409 & 299 & 1\end{pmatrix}^{\intercal}$ & $\mathbf{x}_{3_{\pi_2}} = \begin{pmatrix}416 & 333 & 1\end{pmatrix}^{\intercal}$\\
            4 & $\mathbf{x}_{4_{\pi_1}} = \begin{pmatrix}386 & 022 & 1\end{pmatrix}^{\intercal}$ & $\mathbf{x}_{4_{\pi_2}} = \begin{pmatrix}416 & 000 & 1\end{pmatrix}^{\intercal}$\\
            \hline
        \end{tabular}
        \caption{vecteurs $\mathbf{x}_{i_{\pi_{j}}}$}
        \label{tab_homography_coordinates}
    \end{subtable}
        \caption{Pairs de points pour le calcule de l'Homographie $\mathbf{H}$}
    \end{table}
\begin{remark}
    Les points sur le plan $\pi_2$ sont alignés comme un carré.
\end{remark}
\noindent Les points ont été choisis au début avec l'interface graphique et ont ensuite été maintenus constants pour garantir des résultats reproductibles.
\end{document}
